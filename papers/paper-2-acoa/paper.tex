\documentclass[11pt]{article}
\usepackage[utf-8]{inputenc}
\usepackage[T1]{fontenc}
\usepackage{lmodern}
\usepackage{amsmath,amssymb,amsthm}
\usepackage{hyperref}
\usepackage{microtype}

\title{Autopoietic Coherent Action Orchestration (ACOA): Self-Organizing Autonomous Systems}
\author{MatVerse Research Collective}
\date{}

\begin{document}

\maketitle

\begin{abstract}
This paper introduces ACOA (Autopoietic Coherent Action Orchestration), a self-organizing framework 
where autonomous agents dynamically maintain coherence through adaptive manifold learning. 
We demonstrate homeostatic properties and emergent coordination without explicit supervision.
\end{abstract}

\section{Introduction}

Autopoiesis, the self-maintenance of living systems, inspired our orchestration framework. 
ACOA extends CAS with adaptive learning.

\section{Autopoietic Framework}

\begin{definition}[Autopoietic System]
A system is autopoietic if it autonomously maintains its organization through self-referential processes.
\end{definition}

An ACOA system $(M, \rho, \Phi, \mathcal{L})$ includes:
\begin{itemize}
\item Adaptive learning operator $\mathcal{L}$ that updates $\Phi$ based on coherence history
\item Feedback loop: $\rho_t \to \Phi_t \to M_{t+1} \to \rho_{t+1}$
\end{itemize}

\section{Self-Organization}

Homeostasis condition:
\[\frac{d}{dt}\mathbb{E}[\rho] = 0 \implies \text{self-sustaining coherence}\]

\section{Applications}

\subsection{Biological Inspiration}
ACOA mimics cellular homeostasis at the collective level.

\subsection{Resilience}
Self-organizing systems recover from perturbations autonomously.

\section{Conclusion}

ACOA demonstrates that autonomous orchestration is achievable through adaptive coherence.

\begin{thebibliography}{99}
\bibitem{cas} Building on CAS framework.
\end{thebibliography}

\end{document}
