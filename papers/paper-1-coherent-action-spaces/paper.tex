\documentclass[11pt]{article}
\usepackage[utf-8]{inputenc}
\usepackage[T1]{fontenc}
\usepackage{lmodern}
\usepackage{amsmath,amssymb,amsthm}
\usepackage{hyperref}
\usepackage{microtype}

\title{Coherent Action Spaces: Theoretical Framework and Applications}
\author{MatVerse Research Collective}
\date{}

\begin{document}

\maketitle

\begin{abstract}
We develop the coherent action spaces (CAS) framework, enabling autonomous agents to coordinate 
through shared action manifolds. Applications to multi-agent systems are demonstrated.
\end{abstract}

\section{Introduction}

Building on foundational mathematics, this paper develops the coherent action spaces framework 
for practical autonomous coordination.

\section{Coherent Action Spaces}

\begin{definition}
A coherent action space is a triplet $(M, \rho, \Phi)$ where:
\begin{itemize}
\item $M$ is a differential manifold of feasible actions
\item $\rho$ is a coherence metric on $M \times M$
\item $\Phi: \mathbb{R}^+ \to \mathbb{R}$ is a coherence evolution law
\end{itemize}
\end{definition}

\section{Multi-Agent Coordination}

For $n$ agents with action manifolds $M_1, \ldots, M_n$, the joint coherence is:
\[\rho_{\text{joint}}(a_1, \ldots, a_n) = \prod_{i=1}^{n} \rho_i(a_i)\]

\section{Applications}

\subsection{Swarm Dynamics}
CAS enables emergent swarm coordination without centralized control.

\subsection{Distributed Control}
Agents negotiate coherence through local interactions.

\section{Conclusion}

The CAS framework provides a scalable approach to multi-agent autonomy.

\begin{thebibliography}{99}
\bibitem{foundations} Building on paper-0 foundations.
\end{thebibliography}

\end{document}
