\documentclass[11pt]{article}
\usepackage[utf-8]{inputenc}
\usepackage[T1]{fontenc}
\usepackage{lmodern}
\usepackage{amsmath,amssymb,amsthm}
\usepackage{hyperref}
\usepackage{microtype}

\title{Foundations of Coherent Action Spaces: Mathematical Framework}
\author{MatVerse Research Collective}
\date{}

\begin{document}

\maketitle

\begin{abstract}
This paper establishes the foundational mathematical framework for coherent action spaces (CAS), 
introducing manifold structures and coherence metrics essential for autonomous systems.
\end{abstract}

\section{Introduction}

Coherent action spaces form the theoretical basis for autonomous agent coordination. 
This work develops the underlying mathematics.

\section{Manifold Structures}

Define a manifold $M = \{x \in \mathbb{R}^n : F(x) = 0\}$ with coherence constraints.

\begin{definition}
A coherent action space is a differential manifold equipped with a coherence metric $\rho: M \times M \to \mathbb{R}$.
\end{definition}

\section{Coherence Metrics}

The coherence metric satisfies:
\begin{align}
\rho(x,y) &\geq 0 \\
\rho(x,x) &= 1 \\
\rho(x,y) &= \rho(y,x)
\end{align}

\section{Conclusion}

These foundations enable the development of higher-order action coordination frameworks.

\begin{thebibliography}{99}
\bibitem{arxiv} Example reference for foundational work.
\end{thebibliography}

\end{document}
